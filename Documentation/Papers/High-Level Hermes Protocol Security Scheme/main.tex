\documentclass{article}
\usepackage[margin=15mm]{geometry}
\usepackage{hyperref}
\usepackage[underline=true]{pgf-umlsd}
\usetikzlibrary{calc}
\begin{document}

\title{Hermes Messenger Protocol Description}
\author{Eugene Kovalev and Jonathan Huang}

\maketitle

\section{Introduction}

The following is a high level overview of the initial specification of the Hermes Messenger Protocol. It includes descriptions of how a device establishes a connection with a server, how the user authenticates to the server, and how one user can have a secure conversation with another.

\section{Scope of this Document}

It is important to note that this document is meant to be a high-level overview of the protocol so that others can view the general idea and critique it for security purposes. There are many components that will be part of Hermes, but not mentioned in this paper. This paper will discuss the high-level ideas for a device to begin contact with the server, establishing secured sessions, user authentication to the server, and how two clients can have a conversation.

Some elements that will be added, but not discussed here are group messages, login-time synchronization of messages to a device, media messages, and key-value messages (which will be a key element of synchronizing options and contacts between devices belonging to the same user).

\section{General Overview}

The idea behind the Hermes protocol is to have secure communication between two parties using well-established cryptographic primitives. As such, the protocol begins with a device connecting to a server and confirming that it is indeed the device that it claims to be (via unique public key and a challenge from a server). Once this is established, the server generates a symmetric key to encrypt the rest of the communication for this session. The first thing that must be done in each session is that the client must now confirm their identity as a user (associated with a particular user-name) and then application-level communication can begin.

\subsection{Sessions}

A key element of Hermes is the concept of sessions. Communication begins with asymmetric encryption, but in order to speed things up, a symmetric key is generated as quickly as possible and the rest of the communication is done via a symmetrically encrypted channel. This channel is referred to as a session which is symmetrically encrypted with the session key. Please note that there is a unique session-key generated for each device login. Once the session is over (if either the server or client disconnect) the key is destroyed and the next session must begin with another device login which, if successful, generates a new session key. The symmetric encryption uses the session key which both the server and the client possess and all communication in a session is encrypted with that key.

\subsection{Conversation}

The other key concept in Hermes is a conversation. It is conceptually identical to a session except that instead of happening between a client and a server, it happens between any number of clients. Similar to a session, with the server acting as a middle-man, the clients establish each other's identities and then generate a symmetric key that will be used to encrypt the conversation. The server, because of the fact that the conversation is established via public-key encryption between the clients, cannot possibly know the conversation key. Therefore all conversation-level communication cannot be eavesdropped on, even by the server itself.

\section{Device \& User Authentication}

\begin{center}
  \begin{sequencediagram}
    \newinst{client}{Client}
    \newinst[5]{server}{Server}

    \begin{sdblock}{Establish Session}{Establishes device identity and communication session between the client and server}
      \mess{client}{m}{server}
      \node [anchor=east] (t0) at (mess from) {$m=enc_{Pub_S}(Pub_D)$};
      \node [anchor=west] (t1) at (mess to) {$dec_{Priv_S}(m)=Pub_D$};
      \begin{call}{server}{gen\_challenge()}{server}{$challenge=rand(n)$}
      \end{call}
      \mess{server}{m}{client}
      \node [anchor=west] (t1) at (mess from) {$m=enc_{Pub_D}(challenge)$};
      \node [anchor=east] (t0) at (mess to) {$challenge=dec_{Priv_D}(m)$};
      \mess{client}{m}{server}
      \node [anchor=east] (t0) at (mess from) {$m=sign_{Priv_D}(challenge)$};
      \node [anchor=west] (t1) at (mess to) {$verify_{Pub_D}(m,challenge)$};
      \begin{call}{server}{gen\_session\_key()}{server}{$k_{session}=genKey()$}
      \end{call}
      \mess{server}{m}{client}
      \node [anchor=west] (t1) at (mess from) {$m=enc_{Pub_D}(k_{session},sign_{Priv_S}(k_{session}))$};
      \node [anchor=east] (t1) at (mess to) {$k_{session}, S = dec_{Priv_S}(m)$};
      \mess{client}{m}{server}
      \node [anchor=east] (t0) at (mess from) {$m=verify_{Pub_S}(S,k_{session})$};
    \end{sdblock}

    \begin{sdblock}{Session}{Within a session, all messages are symmetrically encrypted before being sent s.t. $m = enc_{k_{session}}(m)$}
      \begin{sdblock}{Login}{The client must now establish user identify}
        \mess{client}{m}{server}
        \node [anchor=east] (t0) at (mess from) {$m=Pub_U$};
        \node [anchor=west] (t1) at (mess to) {$Pub_U=m$};
        \begin{call}{server}{gen\_challenge()}{server}{$challenge=rand(n)$}
        \end{call}
        \mess{server}{m}{client}
        \node [anchor=west] (t1) at (mess from) {$m=challenge$};
        \node [anchor=east] (t0) at (mess to) {$challenge=m$};
        \mess{client}{m}{server}
        \node [anchor=east] (t0) at (mess from) {$m=sign_{Priv_U}(challenge)$};
        \node [anchor=west] (t1) at (mess to) {$verify_{Pub_U}(m,challenge)$};
        \begin{call}{server}{username($Pub_U$)}{server}{$u=username(Pub_U)$}
        \end{call}
        \mess{server}{m}{client}
        \node [anchor=west] (t1) at (mess from) {$m=u$};
        \node [anchor=east] (t0) at (mess to) {$u=m$};
        \begin{call}{client}{username\_update()}{client}{$u=new\_username()$}
        \end{call}
        \mess{client}{m}{server}
        \node [anchor=east] (t0) at (mess from) {$m=u$};
        \node [anchor=west] (t1) at (mess to) {$u=m$};
      \end{sdblock}
      \begin{sdblock}{Communication}{Communication can now commence}
        \mess{client}{m}{server}
        \mess{server}{m}{client}
      \end{sdblock}
    \end{sdblock}
  \end{sequencediagram}
\end{center}

The diagram above details how the protocol should behave in the establishment of a session. The following definitions should be noted: $enc_{k}(X)$ refers to encryption of a string $X$ with the key $k$ and produces the cipher-text of the string.. If $k$ is $k_{session}$ then the encryption is symmetric and if $k$ is $Pub_Y$ then the encryption is asymmetric. You can also tell whether the encryption is symmetric or asymmetric by looking at which key is used to encrypt/decrypt the message. If the key is the same, then it's symmetric, otherwise it's asymmetric. Similarly, $dec_k(X)$ is decryption and produces the string of a given cipher-text. $rand(n)$ produces a random string of length $n$. $sign_k(X)$ produces the signature of a string $X$ with the key $k$ that is verified with the function $verify_k(X,S)$ where $X$ is the original message and $S$ is the signature. $gen\_key()$ generates a symmetric encryption key. The server-side function $username(k)$ returns the username associated with the key $k$ in the database or $NULL$ is one is not found. On the client-side we have the function $update\_username(u)$ where $u$ is the username sent by the server which updates the username on the client or updates it using the $new\_username()$ function and sends the username (whether new or old) to the server.

\subsection{Establishing a Session}

The first action of a Hermes protocol exchange is to establish a session. This is done when a device (any device, even one that is not registered with the server) first connects with a server. We assert that the device possesses the public key of the server that it is trying to connect to, although this is addressed in more detail below. The device also possesses a public/private key-pair which is generated upon installation of Hermes Messenger, as well as the public/private key of a user which can either be generated (if this is the first device that has Hermes installed for that user) or otherwise transferred to the device.

The device begins by encrypting its public key with the server's public key and sending it to the server. The server, after decrypting it, then generates a challenge - a random string of a particular length, encrypts that with the device's public key, and sends it to the device. The device, in order to confirm that it is indeed the device that it claims to be, and not another that somehow acquired someone else's public key, must decrypt the challenge, sign it, and then encrypt the signature with the server's public key and send it over. Once the server decrypts and verifies the signature, we know that the client is who they claim to be. If there is an issue, the server should terminate the connection. If the device authenticates correctly, the server can then generate a session key, encrypt it with the device key, and send it over to the client along with a signature of the key. The client then verifies that the signature came from the server and sends the server an ``ACK''. That signature part is important, otherwise someone can execute a MITM attack against the system by impersonating the server.

\subsection{Logging In}

Assuming that the device authenticated properly, the client can then log in as a specific user. This is done inside a session using the symmetric key generated earlier when the device authenticated. It is important to note that all messages within a session are symmetrically encrypted with the $k_{session}$ key even though this is not explicitly stated in the diagram above.

The session begins with the client sending over their user's public key. The server receives the key and generates a challenge. The client signs the challenge with the user's public key and sends over the signature which, once verified, confirms that they are indeed the user whose public key they initially transmitted. If the signature is not correct, the server disconnects. If the signature is correct, the server queries the local database to retrieve the username associated with the key. If there is no such username in the database, the result of the query is ``NULL''. The server transmits the result of the query to the client. If the server's response is ``NULL'' then the client must generate a new username, otherwise, they can either adopt the username that the server sent over, or generate a new one by asking the user. Either way, the client sends over their username back to the server and if it is different, the server updates the entry in the database. Once this is done, normal communication can continue between the server and client.

\section{Client-to-Client Communication}

\begin{center}
  \begin{sequencediagram}
    \newinst{a}{Client A}
    \newinst[2]{server}{Server}
    \newinst[2]{b}{Client B}

    \begin{sdblock}{Sessions A \& B}{All messages between either client and the server are encrypted with $k_{session,A}$ and $k_{session,B}$ respectively.}
      \begin{sdblock}{Establish Conversation}{Establishes identity and a conversation between two clients.}
        \mess{a}{m}{server}
        \node [anchor=east] (t0) at (mess from) {$m=Pub_B$};
        \node [anchor=west] (t1) at (mess to) {$Pub_B=m$};
        \mess{server}{m}{b}
        \node [anchor=east] (t1) at (mess from) {$m=Pub_A$};
        \node [anchor=west] (t2) at (mess to) {$Pub_A=m$};
        \mess{b}{}{b}
        \node [anchor=west] (t0) at (mess to) {$c=rand(n)$};
        \mess{b}{m}{a}
        \node [anchor=west] (t2) at (mess from) {$m=enc_{Pub_A}(c,sign_{Priv_B}(c))$};
        \node [anchor=east] (t0) at (mess to) {$c,s=dec_{Priv_A}(m)$};
        \mess{a}{}{a}
        \node [anchor=east] (t0) at (mess to) {$verify_{Pub_B}(s)$};
        \mess{a}{m}{b}
        \node [anchor=east] (t0) at (mess from) {$m=enc_{Pub_B}(sign_{Priv_A}(c))$};
        \node [anchor=west] (t2) at (mess to) {$s=dec_{Priv_B}(m)$};
        \mess{b}{m}{a}
        \node [anchor=west] (t2) at (mess from) {$m=ACK$ ? $verify_{Pub_A}(s)$};
        \node [anchor=east] (t0) at (mess to) {$k_{convo}=gen\_key()$};
        \mess{a}{m}{b}
        \node [anchor=east] (t0) at (mess from) {$m=enc_{Pub_B}(k_{convo},sign_{Priv_A}(k_{convo}))$};
        \node [anchor=west] (t2) at (mess to) {$k_{convo},s=dec_{Priv_B}(m)$};
        \mess{b}{m}{a}
        \node [anchor=west] (t2) at (mess from) {$m=ACK$ ? $verify_{Pub_A}(s)$};
      \end{sdblock}

      \begin{sdblock}{Conversation}{}
        \mess{a}{m}{b}
        \mess{b}{m}{a}
      \end{sdblock}
    \end{sdblock}
  \end{sequencediagram}
\end{center}

Secure client-to-client communication is the primary purpose of Hermes. As such, we use a concept called a ``conversation'' to establish an encrypted channel between two clients. The details of this communication should not be viewable by a server, however, the server can know which clients are communicating with each other. No outside entity should be able to determine which client is talking to another.

\subsection{Reference}

It should be noted, that in the diagram above, while the conversation is being established, then all messages from a client to the server that look like this:

\begin{center}
  \begin{sequencediagram}
    \newinst{a}{Client A}
    \newinst[2]{server}{Server}
    \newinst[2]{b}{Client B}

    \begin{sdblock}{Establish Conversation}{}
      \mess{a}{m}{server}
      \node [anchor=east] (t0) at (mess from) {$m=X$};
      \node [anchor=west] (t1) at (mess to) {$X=m$};
      \mess{server}{m}{b}
      \node [anchor=east] (t1) at (mess from) {$m=Y$};
      \node [anchor=west] (t2) at (mess to) {$Y=m$};
    \end{sdblock}
  \end{sequencediagram}
\end{center}

Are actually this:

\begin{center}
  \begin{sequencediagram}
    \newinst{a}{Client A}
    \newinst[2]{server}{Server}
    \newinst[2]{b}{Client B}

    \begin{sdblock}{Establish Conversation}{}
      \mess{a}{m}{server}
      \node [anchor=east] (t0) at (mess from) {$m=enc_{k_{session,A}}(X)$};
      \node [anchor=west] (t1) at (mess to) {$X=dec_{k_{session,A}}(m)$};
      \mess{server}{m}{b}
      \node [anchor=east] (t1) at (mess from) {$m=enc_{k_{session,B}}(Y)$};
      \node [anchor=west] (t2) at (mess to) {$Y=dec_{k_{session,B}}(m)$};
    \end{sdblock}
  \end{sequencediagram}
\end{center}

And all messages from client to client that look like this:

\begin{center}
  \begin{sequencediagram}
    \newinst{a}{Client A}
    \newinst[2]{server}{Server}
    \newinst[2]{b}{Client B}

    \begin{sdblock}{Establish Conversation}{}
      \mess{a}{m}{b}
      \node [anchor=east] (t0) at (mess from) {$m=X$};
      \node [anchor=west] (t2) at (mess to) {$X=m$};
    \end{sdblock}
  \end{sequencediagram}
\end{center}

Are actually this:

\begin{center}
  \begin{sequencediagram}
    \newinst{a}{Client A}
    \newinst[2]{server}{Server}
    \newinst[2]{b}{Client B}

    \begin{sdblock}{Establish Conversation}{}
      \mess{a}{m}{server}
      \node [anchor=east] (t0) at (mess from) {$m=enc_{k_{session,A}}(X)$};
      \node [anchor=west] (t1) at (mess to) {$X=dec_{k_{session,A}}(m)$};
      \mess{server}{m}{b}
      \node [anchor=east] (t1) at (mess from) {$m=enc_{k_{session,B}}(X)$};
      \node [anchor=west] (t2) at (mess to) {$X=dec_{k_{session,B}}(m)$};
    \end{sdblock}
  \end{sequencediagram}
\end{center}

Once a conversation has been established, then all messages from one client to another that look like this:

\begin{center}
  \begin{sequencediagram}
    \newinst{a}{Client A}
    \newinst[2]{server}{Server}
    \newinst[2]{b}{Client B}

    \begin{sdblock}{Establish Conversation}{}
      \mess{a}{m}{b}
      \node [anchor=east] (t0) at (mess from) {$m=X$};
      \node [anchor=west] (t2) at (mess to) {$X=m$};
    \end{sdblock}
  \end{sequencediagram}
\end{center}

Are actually this:

\begin{center}
  \begin{sequencediagram}
    \newinst{a}{Client A}
    \newinst[2.6]{server}{Server}
    \newinst[2.6]{b}{Client B}

    \begin{sdblock}{Conversation}{}
      \mess{a}{m}{server}
      \node [anchor=east] (t0) at (mess from) {$m=enc_{k_{session,A}}(Pub_B,enc_{convo}(X))$};
      \node [anchor=west] (t1) at (mess to) {$Pub_B,c=dec_{k_{session,A}}(m)$};
      \mess{server}{m}{b}
      \node [anchor=east] (t1) at (mess from) {$m=enc_{k_{session,B}}(Pub_A,c)$};
      \node [anchor=west] (t2) at (mess to) {$Pub_A,c=dec_{k_{session,B}}(m)$};
      \mess{b}{}{b}
      \node [anchor=west] (t2) at (mess to) {$X=dec_{k_{convo}}(c)$};
    \end{sdblock}
  \end{sequencediagram}
\end{center}

\subsection{Establishing a Conversation}

The diagrams above show the high-level and details of establishing a conversation, however, for the sake of clarity, a description should also be included as well. We begin establishing a conversation with one client (A) sending the public key of the client they wish to speak to (B), to the server. The server then sends A's public key to B indicating that A wants to start a conversation. Each of these messages is being sent in a session between A and the server and between B and the server. They are therefore encrypted with the keys that A and B established when they logged in. B then generates a random challenge, signs it, and sends the challenge and signature to A after encrypting it with A's public key. A then receives the challenge along with the signature, verifies the signature, and sends an encrypted signature of the challenge back to B. Once B verifies A's signature, both A and B will have established each other's identities. If either one of them fails verification, the connection should be terminated. Once B verifies A's signature of the challenge, it sends an ``ACK'' and A generates a conversation key $k_{convo}$. A then signs $k_{convo}$ and encrypts it along with the signature to send to B. B receives $k_{convo}$ and the signature. Once the signature is verified, B sends an ``ACK'' and the conversation has been established.

\section{Potential Vulnerabilities}

\subsection{Server Public Keys}

A key question that is not addressed in the paper is: ``How does one securely acquire the server's public key?''. The answer is that the app will be shipped with the keys of the most popular public servers installed on it. Beyond that, anyone hosting their own server will be responsible for securely distributing the keys themselves. It is worth pointing out that each client program must have some sort of public key entry GUI that allows the user to write in, or paste in a public key. It is also worth suggesting that the client have some sort of QR recognition so that keys can be shared that way.

\subsection{Exchanging User Private Keys Between Devices}

Another question that needs to be addressed is how does one exchange user private keys between devices that they own. This should ideally be done by displaying a QR code of the key on the screen of one device, and loading it onto another. An alternative method would be to locally encrypt the key with a password and then upload it, via the server, to the other device where it can be decrypted with the same password. Please note that in order for the second method to be secure, the client must enforce password strength requirements.

\subsection{Malicious Server MITM}

The final and most serious security concern is what if the server itself is running a malicious MITM attack. More specifically, in the scheme described, the server can simply supply the client with a false public key for someone they want to converse with and then MITM all communication between two users. There are two solutions to this and both depend on client implementations. The first is that users need to have the ability to share their public keys via QR codes displayed on their screens. This is the only secure way of sharing keys that completely bypasses the server and cannot be attacked. That being said, sharing contact information via the server is very simple and easy and therefore should be allowed. This means that in the client UI, next to every contact entry, a short signature of their key should always be displayed (i.e. RedArmy90 (CD646621)) which means that people can always visually confirm that the key they possess for a person is the same as the key that person has on their device.

\end{document}
